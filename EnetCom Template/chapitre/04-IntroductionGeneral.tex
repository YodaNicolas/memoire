\chapter*{Introduction générale}
\markboth{\MakeUppercase{Introduction générale}}{}
%\addstarredchapter{INTRODUCTION GÉNÉRALES}
\addcontentsline{toc}{chapter}{Introduction générale}
\adjustmtc
\thispagestyle{MyStyle}


Dans un monde en constante évolution où les frontières entre les savoirs s'estompent et où l'accès à l'information devient primordial, les institutions éducatives se trouvent à un carrefour où l'innovation et l'adaptation sont essentielles. La recherche académique, pilier fondamental de la progression de la connaissance et de la compréhension du monde qui nous entoure se doit d’être doté de nouveaux outils. Le rôle crucial que jouent les universités dans la création, la conservation et la diffusion de cette recherche les contraint à adapter leurs méthodes d'apprentissage et de recherche pour répondre aux besoins croissants de leur communauté universitaire.\par
Il est profondément regrettable de constater qu'au sein de notre université, Norbert ZONGO, l'accès aux précieux résultats des travaux de recherche n'est pas aisé pour sa communauté. La conservation exclusive de ces documents au format physique comporte un risque considérable, tant en raison de leur détérioration progressive au fil du temps que des dangers potentiels d'accidents. Outre cela, il serait déplorable que les auteurs de ces documents soient parfois difficiles à joindre en cas de besoin d’éclaircissements sur divers aspects de leurs travaux ou de suggestions. Cette situation entrave considérablement la progression de leurs recherches et nuit à la fluidité de la communication au sein de la communauté académique. Ce type de situation pourrait se produire dans les cas où le thème de recherche d’un étudiant est la suite des recherches mené par un autre étudiant d’une année antérieure.
Il devient donc essentiel de mettre en place une plateforme facilitant le stockage, l'accès aux documents de thèses et mémoires, et la communication entre les auteurs de ces documents et la communauté universitaire.
Ainsi, notre travail vise à concevoir une bibliothèque numérique pour les thèses et mémoires de l’Université Norbert ZONGO doté d’un moyen d’échange entre les auteurs et la communauté pour une meilleur avancé des recherches.\par
Notre rapport, sous le thème : « Application web des mémoires et thèses soutenue à l'Université Norbert ZONGO » est structuré de la manière suivante :\\
-	Un premier chapitre où nous présenterons l’Université Norbert ZONGO et ses différentes UFR et institut.\\
-	Un second pour la rubrique contexte et problématique.  \\
-	Un troisième chapitre où il sera question des objectifs.\\
-	Un quatrième chapitre où nous présenterons notre méthodologie de travail. Nous expliquerons comment notre travail a été subdivisé en plusieurs étape et comment nous avons évolué du début jusqu’à la fin.\\
-	Un cinquième chapitre qui concerne l’étude conceptuelle où nous examinerons de près les différents besoins en matière de fonctionnalité et où nous déterminerons leur faisabilité.\\
-	Un sixième chapitre dans lequel il sera question de conception. Nous parlerons de la structuration de la base de données des cas et des séquences d’utilisation.\\
-	Un septième chapitre dédié à l’implémentation et la concrétisation. Dans ce chapitre nous évoquerons les différentes étapes de la réalisation de notre travail ainsi que des outils utilisés pour y parvenir.\\
-	Un huitième chapitre qui sera consacré à la présentation de la plateforme. Ici nous présenterons les différentes interfaces de la plateforme.\\
-	Un dernier chapitre qui où nous aborderons l’analyse et la discussion des résultats.
.\par

