\chapter*{INTRODUCTION GENERALE}
\markboth{\MakeUppercase{INTRODUCTION GENERALE}}{}
%\addstarredchapter{INTRODUCTION GÉNÉRALES}
\addcontentsline{toc}{chapter}{INTRODUCTION GENERALE}
\adjustmtc
\thispagestyle{MyStyle}


Dans un monde en constante évolution où les frontières entre les savoirs s'estompent et où l'accès à l'information devient primordial, les institutions éducatives se trouvent à un carrefour où l'innovation et l'adaptation sont essentielles. La recherche académique, pilier fondamental de la progression de la connaissance et de la compréhension du monde qui nous entoure se doit d’être doté de nouveaux outils. Le rôle crucial que jouent les universités dans la création, la conservation et la diffusion de cette recherche les contraint à adapter leurs méthodes d'apprentissage et de recherche pour répondre aux besoins croissants de leur communauté universitaire.\par
Il est profondément regrettable de constater qu'au sein de notre université, Norbert ZONGO, l'accès aux précieux résultats des travaux de recherche n'est pas aisé pour sa communauté. La conservation exclusive de ces documents au format physique comporte un risque considérable, tant en raison de leur détérioration progressive au fil du temps que des dangers potentiels d'accidents. Outre cela, il serait déplorable que les auteurs de ces documents soient parfois difficiles à joindre en cas de besoin d’éclaircissements sur divers aspects de leurs travaux ou de suggestions. Cette situation entrave considérablement la progression de la recherche et nuit à la fluidité de la communication au sein de la communauté académique. Ce type de situation pourrait se produire dans les cas où le thème de recherche d’un étudiant est la suite des recherches menées par un autre étudiant d’une année antérieure.
Il devient donc essentiel de mettre en place une plateforme facilitant le stockage, l'accès aux documents de thèses et mémoires, et la communication entre les membres de la communauté universitaire.
Notre travail vise donc à mettre en place une solution numérique pour faciliter l'accès aux mémoires de thèses et mémoires de master de l’Université Norbert ZONGO doté d’un espace numérique d’échange entre les membres de la communauté universitaire pour une meilleur avancé des travaux de recherches.\par
Le rapport de notre projet, qui a pour thème : «Conception d’une plateforme numérique des mémoires de thèses et mémoires de master soutenus à l’UNZ » est structuré de la manière suivante :\\
-	Un premier chapitre où nous présenterons la structure de formation.\\
-	Un second pour la rubrique contexte, problématique et la méthodologie de travail.  \\

-	Un troisième chapitre où nous présenterons l'étude conceptuelle.\\
-	Un quatrième chapitre qui concerne l’implémentation et la concrétisation. Où nous évoquerons les différentes étapes de la réalisation de notre travail ainsi que des outils utilisés pour y parvenir.\\
-Un cinquième chapitre pour la présentation de la plateforme.\\
-	Un dernier chapitre où nous aborderons l’analyse et la discussion des résultats.
.\par

