\chapter*{REMERCIEMENT}
%\addstarredchapter{REMERCIEMENT}
%\addcontentsline{toc}{chapter}{REMERCIEMENT}
%\adjustmtc
\thispagestyle{MyStyle}

Nous tenons à exprimer notre profonde reconnaissance à notre prestigieuse Université Norbert ZONGO pour l’accueil et les multiples efforts fourni. Ces efforts nous ont permis d’étudier dans la quiétude et la sérénité ce qui a été très utile pour la réalisation de ce projet. 
A l’UFR-ST et son corps enseignant, nous désirons exprimer nos sincères remerciements pour la qualité de la formation reçu depuis le début de notre parcoure. Cette formation nous a permis d’approfondir nos connaissances et d’en acquérir de nouvel qui ont énormément servi lors de la réalisation de ce projet.
    A nos encadrants, nous tenons à exprimer notre plus sincère gratitude pour leur enseignement, leur suivi attentif, leurs précieux conseils, suggestions et leur disponibilité. Leur supervision et suggestions ont considérablement amélioré la qualité de notre travail et ont permis d’atteindre nos objectifs dans des conditions agréable.
Nous souhaitons exprimer notre immense gratitude envers nos familles et nos proches pour leur amour inconditionnel leur encouragement et leur énorme soutient grandissant au fil du temp. Leur présence et inestimable affection ont été une et demeurent une source intarissable de motivation.
Nous adressons aussi nos remerciements à nos amis, pour leur compréhension et leur acceptation malgré nos petits défauts. Être dans un milieu où l’entente, la gaité, la solidarité et l’entraide ne manque pas à été utile pour l’élaboration de ce projet.
Enfin, nous tenons à adresser nos remerciements à toutes ces personnes qui, de près ou de loin ont contribué à la réalisation de ce projet, leur disponibilité et leur collaboration ont été constructifs tout au long de notre parcours.
\par