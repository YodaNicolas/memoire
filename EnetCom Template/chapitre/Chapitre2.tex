\chapter{Contexte, problématique et objectifs}
\markboth{Chapitre 2. Contexte, problématique et objectifs}{} 
\begin{spacing}{1.2}
\minitoc
\thispagestyle{MyStyle}
\end{spacing}
\newpage
\section*{Introduction}
Dans ce chapitre il s'agira pour nous de faire connaitre les fondements de la mise en place de notre plateforme, de faire ressortir la question autour de laquelle gravite notre travail et d'identifier les objectifs que nous nous sommes fixés . \par
\section{Contexte}
La réalisation de notre plateforme numérique s'inscrit dans un contexte où l'accès aux ressources académiques dans les bibliothèques physiques pose de nombreux défis. Hormis de leur richesse, les bibliothèques traditionnelles souffrent souvent de limitations qui entravent l'accès aux informations pour les étudiants et les chercheurs. Parmi ces limitations, on peut citer la disponibilité restreinte des documents, le nombre limité de copies pour des ouvrages particulièrement demandés, et la difficulté à localiser des thèses ou mémoires de qualité et pertinentes.\par
L'accès simultané pour plusieurs personnes à une même ressource est un problème récurrent dans les bibliothèques physiques. Lorsqu'un ouvrage ou une thèse est particulièrement sollicité, les utilisateurs peuvent se retrouver en liste d'attente ou se heurter à l'indisponibilité des documents. Cette situation freine le progrès des recherches et la préparation des travaux académiques.\par
De plus, le manque ou la difficulté à trouver des thèses et mémoires dignes d'intérêt est un autre obstacle majeur. Les utilisateurs doivent souvent consacrer un temps considérable à fouiller parmi des centaines de documents pour identifier ceux qui correspondent le mieux à leurs besoins spécifiques. Ce processus peut être laborieux et décourageant, surtout lorsque les ressources disponibles sont dispersées et mal cataloguées.
\par
\section{Problématique}
Comment surmonter les obstacles liés à l'accès restreint et à la disponibilité limitée des ressources académiques dans les bibliothèques physiques, tout en facilitant la recherche et l'accès simultané aux thèses et mémoires de qualité pour un grand nombre d’utilisateurs ?\par
\section{Objectifs}
Il s'agira pour nous dans cette section de présenter les objectifs fondamentaux d'un tel projet en mettant en lumière son importance dans le contexte de la communauté estudiantine. \par
\subsection{Conservation des thèses et mémoires au format numérique}
La conservation des documents au format numérique représente un tournant majeur dans la gestion et la préservation du patrimoine académique. En optant pour cette approche, la bibliothèque de thèses et mémoires garantit une conservation à long terme des travaux de recherche, tout en offrant une accessibilité accrue aux utilisateurs. Cette transition vers le numérique présente plusieurs avantages significatifs. Elle permettra de :\par

	Préserver l'intégrité des documents dans des conditions optimales. Contrairement aux supports physiques qui sont sujets à la détérioration, à la perte ou au vol, les fichiers numériques peuvent être sauvegardés et stockés de manière sécurisée, réduisant ainsi les risques de perte de données précieuses. De plus, grâce aux technologies de sauvegarde et de stockage en ligne, il est possible de créer des copies de sauvegarde afin de garantir la pérennité des documents, même en cas de sinistre.\par

	Faciliter l’accès pour les utilisateurs. Avec une simple connexion à internet, les étudiants, chercheurs et membres de la communauté académique peuvent consulter et télécharger les documents à tout moment et depuis n'importe quel endroit. Cette facilité d'accès transcende les frontières physiques des bibliothèques traditionnelles, permettant ainsi aux utilisateurs d'explorer un vaste ensemble de ressources documentaires sans contraintes géographiques.\par

	Offrir des fonctionnalités de recherche avancées, facilitant la découverte et l'exploration des documents. Notre plateforme de bibliothèques numériques sera équipée d'outils de recherche sophistiqués, permettant aux utilisateurs de trouver rapidement des documents pertinents en utilisant des mots-clés, des filtres de recherche et d'autres critères de recherche avancés. Cette fonctionnalité de recherche améliorée contribue à optimiser l'efficacité de la recherche académique, en permettant aux utilisateurs de découvrir plus facilement des travaux pertinents dans leur domaine d'intérêt.\par

Notre premier objectif est donc d’assurer la pérennité des documents (résultats des travaux de recherches), faciliter l’accès pour les utilisateurs et offrir des fonctionnalités de recherche avancées. Cette transition vers le numérique représente un pas important vers une gestion plus efficace et une diffusion plus large du savoir académique, contribuant ainsi à l'avancement de la recherche et de l'éducation.
\par

\subsection{Echange entre les auteurs des travaux de recherche et la communauté estudiantine}
Au cœur de la conception de notre plateforme réside un autre objectif primordial, il s'agit de favoriser un échange dynamique et enrichissant entre les auteurs des travaux de recherche et la communauté académique qui les entoure. Cette rubrique se concentre sur la mise en lumière de cet objectif essentiel, soulignant comment notre plateforme aspire à créer un espace d'interaction où les éclaircissements et les discussions entre les auteurs et la communauté peuvent prospérer.

En développant cette plateforme, notre vision est de transcender les frontières traditionnelles de la communication académique en favorisant un dialogue ouvert et transparent. À travers une interface interactive permettant de :\par

	Avoir un espace d’échange où chacun pourra apprécier les travaux ou exposer ses différentes préoccupations afin que les auteurs puissent répondre aux questions, clarifier les points complexes.\par
	Echanger des idées novatrices entre les membres de la communauté universitaire et les auteurs ou même entre auteurs.
	De même, la communauté pourra bénéficier directement des connaissances et des perspectives des auteurs, enrichissant ainsi leur compréhension et leur engagement avec les recherches présentées.\par

En mettant en avant cet objectif central, nous aspirons à créer une plateforme qui transcende les barrières traditionnelles de la recherche académique, favorisant une culture de collaboration, de transparence et d'ouverture. Par cette approche, nous visons à renforcer les liens entre chercheurs et public, à promouvoir une compréhension plus approfondie des travaux de recherche et à catalyser l'innovation et le progrès dans notre communauté universitaire.
\par
Notre second objectif, est de créer un espace d'échange entre les auteurs des travaux de recherche et la communauté académique. En favorisant un dialogue ouvert, transparent et interactif, nous aspirons à créer un environnement propice à la collaboration et à l'enrichissement mutuel des connaissances. À travers cette démarche, nous croyons fermement que notre plateforme contribuera à renforcer les liens au sein de la communauté académique, à promouvoir une culture de partage et d'apprentissage continu, et à catalyser l'avancement de la recherche et de l'innovation.\par
En somme, les objectifs de notre plateforme visent à répondre à un besoin essentiel dans le domaine de la recherche académique : celui de créer un espace dynamique où la numérisation des travaux de recherche et la collaboration entre les auteurs et la communauté universitaire peuvent prospérer. En favorisant la diffusion transparente et accessible et sécurisé des connaissances tout en encourageant les échanges constructifs et la coopération entre les chercheurs et les membres de la communauté, nous aspirons à stimuler l'innovation, à renforcer les liens au sein de la communauté universitaire et à contribuer à l'avancement global de la recherche. Par cette approche intégrée, notre plateforme s'engage à jouer un rôle significatif dans la promotion d'une culture de recherche ouverte, collaborative et tournée vers l'avenir, où le partage du savoir est au cœur de notre mission commune.
\section*{Conclusion}

En conclusion, la mise en place de notre plateforme numérique de thèses et mémoires répond à une nécessité pressante d'améliorer l'accès aux ressources académiques dans un contexte où les bibliothèques physiques présentent de nombreuses limitations. En transformant ces documents en ressources numériques, nous offrons une solution efficace pour surmonter les obstacles liés à la disponibilité restreinte et à l'accès limité aux ouvrages de qualité.\par

Notre démarche s'articule autour de deux objectifs principaux. Le premier est de garantir la conservation à long terme des travaux de recherche tout en facilitant leur accès à un public plus large grâce aux avantages du numérique. Ce passage au format numérique permet non seulement de préserver les documents dans des conditions optimales, mais aussi de les rendre accessibles de partout et à tout moment, tout en intégrant des fonctionnalités de recherche avancées pour optimiser l'efficacité des recherches académiques.\par

Le second objectif vise à favoriser un échange enrichissant entre les auteurs des travaux de recherche et la communauté universitaire. En créant un espace interactif sur notre plateforme, nous encourageons un dialogue ouvert et constructif, permettant aux utilisateurs de poser des questions, d'obtenir des clarifications et d'échanger des idées novatrices. Ce cadre interactif promeut une culture de collaboration et de transparence, essentielle pour le progrès et l'innovation au sein de la communauté académique.\par

Ainsi, notre plateforme aspire à transcender les barrières traditionnelles de la recherche académique, en offrant une solution numérique intégrée qui répond aux besoins de conservation, d'accès et de collaboration.\par