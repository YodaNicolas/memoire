\newpage
\thispagestyle{empty}
\begin{center}
  \renewcommand*{\familydefault}{\defaultFont}
  \fontsize{12pt}{12pt}\selectfont%
  \textbf{
  Conception d'une plateforme numérique\\des mémoires de thèses et mémoires de master soutenus à l'UNZ\\%
  }
\vspace{15pt} {%
  \begin{spacing}{0.05}
    \rule{200pt}{2pt}\\
    \rule{200pt}{0.75pt}\\
  \end{spacing}
  \renewcommand*{\familydefault}{\defaultFont}
  \fontsize{14pt}{14pt}\selectfont%
  \vspace{15pt}
  \textbf{Nicolas YODA}
  \vspace{8pt}
  \begin{spacing}{0.05}
    \rule{200pt}{0.75pt}\\
    \rule{200pt}{2pt}\\
  \end{spacing}
}
\end{center}

%Français
\fontsize{12pt}{12pt}\selectfont%
\underline{\textbf{Résumé:}}\\
La réalisation de notre bibliothèque numérique s'inscrit dans un contexte où l'accès aux ressources académiques dans les bibliothèques physiques présente de nombreux défis. Les limitations de disponibilité des documents, le nombre limité de copies, et la difficulté à localiser des thèses ou mémoires de qualité et pertinents sont autant de problématiques que nous avons cherché à surmonter. Cette plateforme vise à offrir une solution moderne et efficace pour centraliser et faciliter l'accès aux documents académiques.

Notre projet s'est structuré autour de plusieurs étapes clés, de la conception à la mise en œuvre. Dans ce mémoire, nous détaillons le processus de développement, en commençant par le choix des technologies appropriées. HTML et CSS ont été utilisés pour créer une interface utilisateur intuitive et responsive, tandis que PHP avec le framework Laravel et JavaScript avec la bibliothèque jQuery ont fourni la robustesse nécessaire pour les fonctionnalités de backend et d'interactivité.

\begin{spacing}{1}
\underline{\textbf{Mots clés:}}Backend, Framework, Responsive, Bibliothèque jQuery.\\
\end{spacing}


\underline{\textbf{Abstract :}}\\
The creation of our digital library addresses the challenges associated with accessing academic resources in physical libraries. Issues such as limited document availability, a restricted number of copies, and the difficulty in locating high-quality and relevant theses or dissertations were significant obstacles we aimed to overcome. This platform aims to offer a modern and efficient solution for centralizing and facilitating access to academic documents.

Our project followed several key stages, from design to implementation. In this thesis, we detail the development process, starting with the selection of appropriate technologies. HTML and CSS were used to create an intuitive and responsive user interface, while PHP with the Laravel framework and JavaScript with the jQuery library provided the necessary robustness for backend functionalities and interactivity.\par
\begin{spacing}{1}
\underline{\textbf{Key-words:}} backend, Framework, Responsive, jQuery library.\\
\end{spacing}



    